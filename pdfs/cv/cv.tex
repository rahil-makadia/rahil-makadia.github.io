% Inspired by: https://github.com/jitinnair1/autocv
%----------------------------------------------------------------------------------------
%	DOCUMENT DEFINITION
%----------------------------------------------------------------------------------------

% article class because we want to fully customize the page and not use a cv template
\documentclass[letterpaper,12pt]{article}

%----------------------------------------------------------------------------------------
%	FONT
%----------------------------------------------------------------------------------------

% % fontspec allows you to use TTF/OTF fonts directly
% \usepackage{fontspec}
% \defaultfontfeatures{Ligatures=TeX}

% % modified for ShareLaTeX use
% \setmainfont[
% SmallCapsFont = Fontin-SmallCaps.otf,
% BoldFont = Fontin-Bold.otf,
% ItalicFont = Fontin-Italic.otf
% ]
% {Fontin.otf}

%----------------------------------------------------------------------------------------
%	PACKAGES
%----------------------------------------------------------------------------------------
\usepackage{url}
\usepackage{parskip} 	

%other packages for formatting
\RequirePackage{color}
\RequirePackage{graphicx}
\usepackage[usenames,dvipsnames]{xcolor}
\usepackage[scale=0.9]{geometry}

%tabularx environment
\usepackage{tabularx}

%for lists within experience section
\usepackage{enumitem}

% centered version of 'X' col. type
\newcolumntype{C}{>{\centering\arraybackslash}X} 

%to prevent spillover of tabular into next pages
\usepackage{supertabular}
\usepackage{tabularx}
\newlength{\fullcollw}
\setlength{\fullcollw}{0.47\textwidth}

%custom \section
\usepackage{titlesec}				
\usepackage{multicol}
\usepackage{multirow}

%CV Sections inspired by: 
%http://stefano.italians.nl/archives/26
\titleformat{\section}{\Large\scshape\raggedright}{}{0em}{}[\titlerule]
\titlespacing{\section}{0pt}{10pt}{10pt}

%for publications
\usepackage[backend=biber,style=phys,sorting=none,maxbibnames=3]{biblatex}

%Setup hyperref package, and colours for links
\usepackage[unicode, draft=false]{hyperref}
\definecolor{linkcolour}{rgb}{0,0.2,0.6}
\hypersetup{colorlinks,breaklinks,urlcolor=linkcolour,linkcolor=linkcolour}
\addbibresource{pub_journal.bib}
\addbibresource{pub_conf.bib}
\setlength\bibitemsep{1em}

%for social icons
\usepackage{fontawesome5}

%flag for digital CV
\newif\ifdigital
\digitaltrue

%----------------------------------------------------------------------------------------
%	BEGIN DOCUMENT
%----------------------------------------------------------------------------------------
\begin{document}

% non-numbered pages
\pagestyle{empty} 

%----------------------------------------------------------------------------------------
%	TITLE
%----------------------------------------------------------------------------------------

\begin{tabularx}{\linewidth}{@{} C @{}}
\Huge{Rahil Makadia} \\[7.5pt]
\ifdigital
\href{mailto:makadia2@illinois.edu}{\raisebox{-0.05\height}\faEnvelope \ makadia2@illinois.edu} \ $|$ \ 
\href{https://linkedin.com/in/rahil-makadia}{\raisebox{-0.05\height}\faLinkedin\ LinkedIn} \ $|$ \ 
\href{https://rahil-makadia.github.io}{\raisebox{-0.05\height}\faGlobe \ Website} \ $|$ \ 
\href{https://github.com/rahil-makadia}{\raisebox{-0.05\height}\faGithub\ GitHub} \\
\else
\href{mailto:makadia2@illinois.edu}{\raisebox{-0.05\height}\faEnvelope \ makadia2@illinois.edu} \ $|$ \ 
307 Talbot Laboratory, 104 S Wright St, Urbana IL 61801, USA \\
\fi
\end{tabularx}

%----------------------------------------------------------------------------------------
%	EDUCATION
%----------------------------------------------------------------------------------------
\section{Education}
\begin{tabularx}{\linewidth}{@{}l X@{}}
Ph.D. in Aerospace Engineering, University of Illinois at Urbana-Champaign & \hfill Jan 2021 - present \\
B.S. in Aerospace Engineering, University of Illinois at Urbana-Champaign & \hfill Aug 2017 - Dec 2020 \\
\end{tabularx}

%----------------------------------------------------------------------------------------
% EXPERIENCE SECTIONS
%----------------------------------------------------------------------------------------

% Work Experience
\section{Work Experience}
\begin{tabularx}{\linewidth}{ @{}l r@{} }
\textbf{NASA Jet Propulsion Laboratory} & \hfill Advisors: Steven Chesley, Davide Farnocchia \\[3.75pt]
NSTGRO Visiting Technologist & \hfill May 2023 - Aug 2023 \\[3.75pt]
    \multicolumn{2}{@{}X@{}}{
        \begin{minipage}[t]{\linewidth}
            \begin{itemize}[nosep,after=\strut, leftmargin=1em, itemsep=3pt]
                \item[--] Validated an efficient solar system propagator with sub\,-1\,km position accuracy over 250 years compared to JPL's internal small body software.
                \item[--] Developed an orbit determination submodule alongside the propagator that has sub\,-\,1$\sigma$ agreement with JPL small-body orbit solutions.
                \item[--] Presented a publicly available Python package, \textsc{grss}, to allow the scientific community to accurately propagate and determine the orbits of solar system bodies. 
            \end{itemize}
        \end{minipage}
    }
\end{tabularx}

\begin{tabularx}{\linewidth}{ @{}l r@{} }
\textbf{NASA Goddard Space Flight Center} & \hfill Advisors: Kenneth Getzandanner, Andrew Liounis \\[3.75pt]
OSIRIS-REx/OSIRIS-APEX Celestial Navigation Intern & \hfill Jun 2022 - Aug 2022 \\[3.75pt]
    \multicolumn{2}{@{}X@{}}{
        \begin{minipage}[t]{\linewidth}
            \begin{itemize}[nosep,after=\strut, leftmargin=1em, itemsep=3pt]
                \item[--] Developed simulations to assess performance of celestial navigation using onboard optical instruments during the OSIRIS-APEX cruise phase.
                \item[--] Simulated more than 8,000 solar system bodies to obtain optimal observable clusters for the spacecraft.
                \item[--] Performed covariance analysis using NASA's MONTE software to study the spacecraft's state uncertainty on its way to asteroid (99942) Apophis.
            \end{itemize}
        \end{minipage}
    }
\end{tabularx}

% Research Experience
\section{Research Experience}
\begin{tabularx}{\linewidth}{ @{}l r@{} }
\textbf{Astrodynamics and Planetary Exploration Research Group} & \hfill Advisor: Dr. Siegfried Eggl \\[3.75pt]
Double Asteroid Redirection Test (DART) Mission & \hfill May 2020 - present \\[3.75pt]
    \multicolumn{2}{@{}X@{}}{
        \begin{minipage}[t]{\linewidth}
            \begin{itemize}[nosep,after=\strut, leftmargin=1em, itemsep=3pt]
                \item[--] Analyzed high-fidelity kinetic impactor simulation results from NASA's Jet Propulsion Laboratory (JPL) for impacts in the (65803) Didymos binary asteroid system.
                % \item[--] Simulated the effect of DART spacecraft's impact on Dimorphos and the change in the system's heliocentric orbit.
                \item[--] Implemented a novel method to impart momentum changes in the Didymos system after the DART impact.
                \item[--] Produced a post-deflection impact risk assessment for Didymos using parallelized Monte Carlo simulations.
                \item[--] Generated updated B-plane maps to conclude that the Didymos system will not collide with the Earth after the DART impact.
                % \item[--] Simulated expected post-DART radio and optical observations of the Didymos system for covariance analysis.
                \item[--] Wrote MATLAB and Python parameter estimation packages to assess measurability of the heliocentric momentum enhancement from the DART impact.
            \end{itemize}
        \end{minipage}
    }
\end{tabularx}

\begin{tabularx}{\linewidth}{ @{}l r@{} }
    Gauss-Radau Small-body Simulator (GRSS) & \hfill Nov 2022 - present \\[3.75pt]
        \multicolumn{2}{@{}X@{}}{
            \begin{minipage}[t]{\linewidth}
                \begin{itemize}[nosep,after=\strut, leftmargin=1em, itemsep=3pt]
                    \item[--] Released an open-source Python library with a C++ binding for use by the planetary defense community.
                    \item[--] Developed a high-accuracy propagator for solar system bodies using Gauss-Radau quadrature.
                    \item[--] Built an orbit determination code for estimating small body orbits using optical and radar observations.
                \end{itemize}
            \end{minipage}
        }
\end{tabularx}

\begin{tabularx}{\linewidth}{ @{}l r@{} }
\textbf{Aerospace Mission Analysis Laboratory} & \hfill Advisor: Dr. Zachary Putnam \\[3.75pt]
Venus Aerogravity Assist Performance Assessment & \hfill Aug 2022 - Jan 2023 \\[3.75pt]
    \multicolumn{2}{@{}X@{}}{
        \begin{minipage}[t]{\linewidth}
            \begin{itemize}[nosep,after=\strut, leftmargin=1em, itemsep=3pt]
                \item[--] Analyzed Venus aerogravity assist missions to significantly reduce transit times to the outer solar system.
                \item[--] Assessed the performance of blunt-body vehicles and waveriders using a MATLAB pipeline for varying trajectories and vehicle configurations.
            \end{itemize}
        \end{minipage}
    }
\end{tabularx}

% Teaching Experience
\section{Teaching Experience}
\begin{tabularx}{\linewidth}{ @{}l r@{} }
\textbf{University of Illinois at Urbana-Champaign} & \hfill Instructor: Dr. Siegfried Eggl \\[3.75pt]
Teaching Assistant for AE 352: Aerospace Dynamical Systems & \hfill Aug 2021 - Dec 2021 \\[3.75pt]
    \multicolumn{2}{@{}X@{}}{
        \begin{minipage}[t]{\linewidth}
            \begin{itemize}[nosep,after=\strut, leftmargin=1em, itemsep=3pt]
                \item[--] Focused on developing and teaching the curriculum's core dynamics course with aerospace applications.
                \item[--] Syllabus emphasized on covering Newtonian, Lagrangian, and Hamiltonian mechanics to represent particle motion.
                \item[--] Assisted 16 student teams with Project Clear Constellation, which called for novel methods to remove orbital debris.
            \end{itemize}
        \end{minipage}
    }
\end{tabularx}

% \begin{tabularx}{\linewidth}{ @{}l r@{} }
%     \textbf{University of Illinois at Urbana-Champaign} & \hfill Instructor: Dr. Huy Tran \\[3.75pt]
%     Undergraduate Course Assistant for AE 199: Aerospace Computing & \hfill Jan 2020 - May 2020 \\[3.75pt]
%         \multicolumn{2}{@{}X@{}}{
%             \begin{minipage}[t]{\linewidth}
%                 \begin{itemize}[nosep,after=\strut, leftmargin=1em, itemsep=3pt]
%                     \item[--] Assisted with grading for a new course focused on using Python to solve problems such as analyzing air traffic data and designing Martian landers.
%                     \item[--] Worked with instructor to augment course for a fully online learning environment without affecting students due to the COVID-19 pandemic.
%                 \end{itemize}
%             \end{minipage}
%         }
% \end{tabularx}

%----------------------------------------------------------------------------------------
%	SKILLS
%----------------------------------------------------------------------------------------
\section{Skills}
\begin{tabularx}{\linewidth}{@{}l X@{}}
Programming Languages  &  Python, C/C++, MATLAB, Fortran \\[3.75pt]
Software Tools  &  \LaTeX, Git \\[3.75pt]
Prepackaged Tools  &  MONTE, GMAT, FreeFlyer \\[3.75pt]
Operating Systems  &  Linux, MacOS, Windows \\[3.75pt]
Languages  &  English, Gujarati, Hindi, French \\[3.75pt]
\end{tabularx}

%----------------------------------------------------------------------------------------
%	AWARDS/AFFILIATIONS
%----------------------------------------------------------------------------------------
\section{Awards and Affiliations}
\begin{tabularx}{\linewidth}{@{}l X@{}}
\textbf{NASA Space Technology Graduate Research Opportunities Fellow} & \hfill Aug 2022 - present \\[3.75pt]
NSTGRO award from NASA Space Technology Mission Directorate (STMD) & \hfill \\[3.75pt]
\end{tabularx}
\begin{tabularx}{\linewidth}{@{}l X@{}}
\textbf{Achievement Rewards for College Scientists (ARCS) Scholar} & \hfill Aug 2023 - present \\[3.75pt]
ARCS Foundation Illinois Chapter & \hfill \\[3.75pt]
\end{tabularx}
\begin{tabularx}{\linewidth}{@{}l X@{}}
\textbf{Double Asteroid Redirection Test (DART) Investigation Team Member} & \hfill Dec 2020 - Nov 2023 \\[3.75pt]
NASA/Johns Hopkins University Applied Physics Laboratory (APL) & \hfill \\[3.75pt]
\end{tabularx}
\begin{tabularx}{\linewidth}{@{}l X@{}}
\textbf{John C. Mather Nobel Scholar} & \hfill Jul 2022 - Jun 2023 \\[3.75pt]
National Space Grant Foundation & \hfill \\[3.75pt]
\end{tabularx}
\begin{tabularx}{\linewidth}{@{}l X@{}}
\textbf{Fellowship for Outstanding Academic and Research Achievement} & \hfill Apr 2023 \\[3.75pt]
Aerospace Engineering Department at the University of Illinois at Urbana-Champaign & \hfill \\[3.75pt]
\end{tabularx}
\begin{tabularx}{\linewidth}{@{}l X@{}}
\textbf{President's Award} & \hfill Aug 2017 - Dec 2020 \\[3.75pt]
University of Illinois at Urbana-Champaign & \hfill \\[3.75pt]
\end{tabularx}
\begin{tabularx}{\linewidth}{@{}l X@{}}
\textbf{Hans von Muldau Team Award for Best Team Project} & \hfill October 2019 \\[3.75pt]
70\textsuperscript{th} International Astronautical Congress (IAC), Washington D.C. & \hfill \\[3.75pt]
\end{tabularx}
\begin{tabularx}{\linewidth}{@{}l X@{}}
\textbf{Dean's List} & \hfill Spring 2019, Spring 2020 \\[3.75pt]
University of Illinois at Urbana-Champaign & \hfill \\[3.75pt]
\end{tabularx}

%----------------------------------------------------------------------------------------
%	PUBLICATIONS
%----------------------------------------------------------------------------------------
\newpage
\section{Journal Articles}
\begin{refsection}[pub_journal.bib]
\nocite{*}
\printbibliography[heading=none]
\end{refsection}

\section{Conference and Meeting Proceedings}
\begin{refsection}[pub_conf.bib]
\nocite{*}
\printbibliography[heading=none]
\end{refsection}

%----------------------------------------------------------------------------------------
%	LAST UPDATED
%----------------------------------------------------------------------------------------
\vfill
\center{\footnotesize Last updated: \today}

\end{document}
